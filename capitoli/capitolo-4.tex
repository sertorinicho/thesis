% !TEX encoding = UTF-8
% !TEX TS-program = pdflatex
% !TEX root = ../tesi.tex

%**************************************************************
\chapter{Metodologie}
\label{cap:metodologie}
%**************************************************************

%\intro{Breve introduzione al capitolo}\\

%**************************************************************
\section{Strumenti e Tecnologie utilizzate}
Per poter eseguire il confronto tra le prestazioni di database relazionali e NoSQL è stato necessario preparare tutta una serie di strumenti utili a far funzionare i database stessi e al monitoraggio dei dati contenuti al loro interno.

\subsection{Docker}
Per poter effettuare test consistenti, evitare problemi di installazione e compartimentalizzare l'ambiente di lavoro, si è deciso di sfruttare Docker. Questo software permette di virtualizzare l'esecuzione di altri applicativi in ambienti chiusi e controllati, facilitando determinati processi di sviluppo. L'utilizzo di Docker in questo progetto lo rende più maneggevole e lineare.\\
Per prima cosa è quindi necessario installare l'engine di Docker che ci permette di gestire i vari container che andremo a creare.\\
Nello specifico, un container è una istanziazione di un'immagine, che a sua volta è uno snapshot del software che vogliamo "dockerizzare".\\
Docker ci permette di creare più container che funzionano parallelamente e indipendentemente gli uni dagli altri. Possono tuttavia comunicare utilizzando delle porte specificate nella fase di creazione dell'immagine.\\
Nel caso del progetto in corso, questo torna utile innanzitutto per gestire il funzionamento dei due DB (che saranno PostgreSQL e MongoDB), inserendoli entrambi in un rispettivo container.
Per facilitare poi i processi di comunicazione con i DB verranno utilizzati altri due container contenenti delle interfacce grafiche (rfispettivamente pgAdmin e MongoExpress).

\subsection{PostgreSQL e tecnologie annesse}

\subsection{MongoDB e tecnologie annesse}

%**************************************************************
\section{Selezione di un ambito di confronto}

%**************************************************************
\section{Modellazione del database in MongoDB}