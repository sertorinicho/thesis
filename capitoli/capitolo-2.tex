% !TEX encoding = UTF-8
% !TEX TS-program = pdflatex
% !TEX root = ../tesi.tex

%**************************************************************
\chapter{Descrizione dello Stage}
\label{cap:descrizione-stage}
%**************************************************************

%\intro{Brevissima introduzione al capitolo}\\

%**************************************************************
\section{Introduzione al Progetto}

L'obiettivo dello stage è quello di effettuare uno studio preliminare delle tecnologie che gravitano attorno al concetto di NoSQL, per evidenziarne vantaggi e svantaggi, in modo da poter valutare in maniera concreta eventuali possibilità di integrazione nello stack aziendale. \\
Una volta portato a termine questo studio, si vuole portare a confronto le soluzioni attualmente adottate con quelle analizzate, per valutare in modo concreto in che modo queste ultime possono portare ad un miglioramento nella gestione e nell'utilizzo degli applicativi aziendali.

%**************************************************************
\section{Obiettivi Formativi}

Gli obiettivi formativi dell'attività di stage sono i seguenti:
\begin{itemize}
    \item Approfondire conoscenze in ambito NoSQL;
    \item Apprendere come effettuare attività di test di carico per la valutazione prestazionale di un Database;
    \item Apprendere metodologie ed approcci propri dell'ambiente lavorativo, diversi da quelli universitari.
\end{itemize}

%**************************************************************
\section{Attività preventivate}

La durata complessiva dello stage è stata di 8 settimane di lavoro a tempo pieno per un totale di circa trecentoventi ore.\\
\noindent Secondo il piano di lavoro iniziale definito con l'azienda, le attività sono distribuite come riportato in \autoref{tab:pianificazione}.\\

%\begin{comment}

\begin{center}

        \renewcommand{\arraystretch}{1.5}
    
        \centering
        \begin{longtable}{| C{2.5cm} | C{2cm} | L{7.2cm} | }
            
            \hline
            
            \rowcolor{mongogreen}
            \textbf{Durata in ore} & \textbf{Settimana} & \textbf{Descrizione} \\
            
            \hline
            
            40 & 1 &
            \begin{itemize}[leftmargin=*]
                \item Formazione sullo stack operativo e di sviluppo aziendale;
                \item Formazione stack Java EE
            \end{itemize} \\
            
            \hline
            
            80 & 2, 3 &
            \begin{itemize}[leftmargin=*]
                \item Studio NoSQL in generale;
                \item Verifica soluzioni NoSQL specifiche;
                \item Identificazione di soluzioni NoSQL enterprise da analizzare e KPI aziendali. 
            \end{itemize}  \\
            
            \hline
        
            
            80 & 4, 5 &
            \begin{itemize}[leftmargin=*]
                \item Analisi di dettaglio delle soluzioni precedentemente identificate. 
            \end{itemize}  \\
             
            \hline
            
            80 & 6, 7 &
            \begin{itemize}[leftmargin=*]
                \item Creazione e codifica di test per lo stuio delle performance di carico e aderenza ai KPI individuati.
            \end{itemize}  \\
            
            \hline
            
            40 & 8 &
            \begin{itemize}[leftmargin=*]
                \item Revisione test e stesura documentazione finale.
            \end{itemize} \\
            
            \hline
            
            \rowcolor{mongogreen}
            \multicolumn{2}{ | c | }{\textbf{Totale ore: }} &   \multicolumn{1}{  c | }{\textbf{320}}\\
            
            \hline
        
            
            \caption{Pianificazione delle attività}\label{tab:pianificazione}
        \end{longtable}
        
    
\end{center}

%\end{comment}

Durante il periodo di stage in azienda il piano di lavoro ha subito modifiche e di conseguenza il consuntivo delle attività svolte riportato nel capitolo conclusivo diverge dalla pianificazione qui presentata. Tali modifiche sono state effettuate in risposta al naturale evolversi del progetto di fronte ad imprevisti ed esigenze nate durante il percorso. \\
Durante tutta la durata del tirocinio si è lavorato a contatto con il relatore preposto all'interno dell'azienda e con varie altre figure di riferimento più esperte nei vari ambiti toccati dal progetto.