% !TEX encoding = UTF-8
% !TEX TS-program = pdflatex
% !TEX root = ../tesi.tex

%**************************************************************
\chapter{Descrizione dello Stage}
\label{cap:descrizione-stage}
%**************************************************************

%\intro{Brevissima introduzione al capitolo}\\

%**************************************************************
\section{Introduzione al progetto}

L'obiettivo dello stage è quello di effettuare uno studio preliminare delle tecnologie che gravitano attorno al concetto di NoSQL, per evidenziarne vantaggi e svantaggi, in modo da poter valutare in maniera concreta eventuali possibilità di integrazione nello stack aziendale. \\
Una volta portato a termine questo studio, si vuole mettere a confronto le soluzioni attualmente adottate con quelle analizzate, per valutare in modo concreto in che modo queste ultime possono portare ad un miglioramento nella gestione e nell'utilizzo degli applicativi aziendali.

%**************************************************************
\section{Obiettivi formativi}

\subsection{Notazione}
Si farà riferimento ai requisiti secondo le seguenti notazioni:
\begin{itemize}
    \item O per i requisiti obbligatori, vincolanti in quanto obiettivo primario richiesto dal committente;
    \item D per i requisiti desiderabili, non vincolanti o strettamente necessari, ma dal riconoscibile valore aggiunto;
    \item F per i requisiti facoltativi, rappresentanti valore aggiunto non strettamente competitivo.
\end{itemize}

\noindent Le sigle precedentemente indicate saranno seguite da una coppia sequenziale di numeri, identificativo del requisito.

\subsection{Obiettivi fissati}
Si prevede il raggiungimento dei seguenti obiettivi:
\begin{itemize}
    \item Obbligatori
    \begin{itemize}
        \item \texttt{O01}: Comprendere lo stack di sviluppo aziendale ed evidenziare eventuali lacune che devono essere colmate con l’ausilio del tutor nelle settimane successive;
        \item \texttt{O02}: Colmare eventuali lacune riguardanti lo stack Java EE;
        \item \texttt{O03}: Colmare eventuali lacune in ambito NoSQL;
        \item \texttt{O04}: Identificare i KPI aziendali;
        \item \texttt{O05}: Ottenere una visualizzazione chiara del panorama NoSQL attuale e dei compromessi a cui ciascuna soluzione inevitabilmente è legata;
        \item \texttt{O06}: Suite di test di carico e verifica performance e KPI;
        \item \texttt{O07}: Esecuzione test e stesura documentazione riassuntiva.
    \end{itemize}
    \item Desiderabili
    \begin{itemize}
        \item \texttt{D01}: Identificazione librerie Java di connessione.
    \end{itemize}
    \item Facoltativi
    \begin{itemize}
        \item Non sono stati individuati obiettivi facoltativi.
    \end{itemize}
\end{itemize}

%**************************************************************
\section{Attività preventivate}

La durata complessiva dello stage è stata di 8 settimane di lavoro a tempo pieno per un totale di circa trecentoventi ore.\\
\noindent Secondo il piano di lavoro iniziale definito con l'azienda, le attività sono distribuite come riportato in \autoref{tab:pianificazione}.\\

\begin{table}
\begin{center}

        \renewcommand{\arraystretch}{1.5}
    
        \centering
        \begin{longtable}{| C{2.5cm} | C{2cm} | L{7.2cm} | }
            
            \hline
            
            \rowcolor{mongogreen}
            \textbf{Durata in ore} & \textbf{Settimana} & \textbf{Descrizione} \\
            
            \hline
            
            40 & 1 &
            \begin{itemize}[leftmargin=*]
                \item Formazione sullo stack operativo e di sviluppo aziendale;
                \item Formazione stack \textit{Java EE}
            \end{itemize} \\
            
            \hline
            
            80 & 2, 3 &
            \begin{itemize}[leftmargin=*]
                \item Studio NoSQL in generale;
                \item Verifica soluzioni NoSQL specifiche;
                \item Identificazione di soluzioni NoSQL enterprise da analizzare e \gls{KPI}\ped{G} aziendali. 
            \end{itemize}  \\
            
            \hline
        
            
            80 & 4, 5 &
            \begin{itemize}[leftmargin=*]
                \item Analisi di dettaglio delle soluzioni precedentemente identificate. 
            \end{itemize}  \\
             
            \hline
            
            80 & 6, 7 &
            \begin{itemize}[leftmargin=*]
                \item Creazione e codifica di test per lo studio delle performance di carico e aderenza ai \gls{KPI}\ped{G} individuati.
            \end{itemize}  \\
            
            \hline
            
            40 & 8 &
            \begin{itemize}[leftmargin=*]
                \item Revisione test e stesura documentazione finale.
            \end{itemize} \\
            
            \hline
            
            \rowcolor{mongogreen}
            \multicolumn{2}{ | c | }{\textbf{Totale ore: }} &   \multicolumn{1}{  c | }{\textbf{320}}\\
            
            \hline
        
            
            \caption{Pianificazione delle attività}\label{tab:pianificazione}
        \end{longtable}
        
    
\end{center}
\end{table}

\noindent Durante il periodo di stage in azienda, il lavoro svolto ha subito leggere variazioni rispetto a quanto riportato nel piano iniziale, e di conseguenza le attività svolte divergono leggermente dalla pianificazione qui presentata, come riporteremo nella valutazione retrospettiva riportata al \autoref{cap:conclusioni}. Le variazioni sono state effettuate in risposta al naturale evolversi del progetto di fronte ad imprevisti ed esigenze nate durante il percorso. \\
Durante tutta la durata del tirocinio si è lavorato a contatto con il relatore preposto all'interno dell'azienda e con varie altre figure di riferimento più esperte nei vari ambiti toccati dal progetto.

