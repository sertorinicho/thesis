% !TEX encoding = UTF-8
% !TEX TS-program = pdflatex
% !TEX root = ../tesi.tex

%**************************************************************
\chapter{Descrizione dello Stage}
\label{cap:capitolo2}
%**************************************************************

%\intro{Brevissima introduzione al capitolo}\\

%**************************************************************
\section{Introduzione al Progetto}

L'obiettivo dello stage è quello di effettuare uno studio preliminare delle tecnologie che gravitano attorno al concetto di NoSQL, per evidenziarne vantaggi e svantaggi, in modo da poter valutare in maniera concreta eventuali possibilità di integrazione nello stack aziendale. \\
Una volta portato a termine questo studio, si vuole portare a confronto le soluzioni attualmente adottate con quelle analizzate, per valutare in modo concreto in che modo queste ultime possono portare ad un miglioramento nella gestione e nell'utilizzo degli applicativi aziendali.

%**************************************************************
\section{Obiettivi Formativi}

Gli obiettivi formativi dell'attività di stage sono i seguenti:
\begin{itemize}
    \item Approfondire conoscenze in ambito NoSQL;
    \item Apprendere come effettuare attività di test di carico per la valutazione prestazionale di un Database;
    \item Apprendere metodologie ed approcci propri dell'ambiente lavorativo, diversi da quelli universitari.
\end{itemize}

%**************************************************************
\section{Attività preventivate}

La durata complessiva dello stage è stata di 8 settimane di lavoro a tempo pieno per un totale di circa trecentoventi ore.\\

Secondo il piano di lavoro iniziale definito con l'azienda, le attività sono distribuite come riportato in tabella.\\

% tabella
Durante il periodo di stage in azienda il piano di lavoro ha subito modifiche e di conseguenza il consuntivo delle attività svolte riportato nel capitolo conclusivo diverge dalla pianificazione qui presentata. Tali modifiche sono state effettuate in risposta al naturale evolversi del progetto di fronte ad imprevisti ed esigenze nate durante il percorso. \\
Durante tutta la durata del tirocinio si è lavorato a contatto con il relatore preposto all'interno dell'azienda e con varie altre figure di riferimento più esperte nei vari ambiti toccati dal progetto.