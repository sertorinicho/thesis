% !TEX encoding = UTF-8
% !TEX TS-program = pdflatex
% !TEX root = ../tesi.tex

%**************************************************************
\chapter{Conclusioni}
\label{cap:conclusioni}
%**************************************************************
\section{Risultati generali del confronto tra database}
Prima di avviare questo progetto di tesi è stata effettuata una breve ricerca per comprendere se l'argomento fosse stato già esplorato in precedenza in altri contesti.\\
Sono state individuate altre tesi dai temi simili che propongono il confronto all'interno di database ``da laboratorio'', e alcune ricerche condotte su larga scala da aziende.\\
Vista la natura di questo progetto, legato ai prodotti software sviluppati da Ifin Sistemi, la prospettiva di effettuare una ricerca approfondita risultava comunque interessante.\\

\noindent Le conclusioni raggiunte al termine dei due mesi di tirocinio non sono dissimili da quelle raggiunte da altre ricerche, come ad esempio quelle svolte da OnGres\cite{site:ongres}, quando queste sono state condotte senza lo scopo di pubblicizzare l'una o l'altra tecnologia.\\
Come in molti altri ambiti, anche in questo caso il contesto di utilizzo di uno strumento ne determina efficacia ed efficienza.\\
L'introduzione dei database NoSQL ha permesso di raggiungere risultati estremamente vantaggiosi in molti ambiti applicativi, ma questo non li rende uno strumento universale per potenziare qualsiasi database in ogni contesto.\\
Tale fatto è valido per qualsiasi tecnologia, ma lo è ancora di più quando si parla di database NoSQL, data la grande varietà di soluzioni che questo gruppo comprende.\\
Nel caso specifico di MongoDB è facile essere tentati di farne largo uso in molti contesti diversi, grazie alla sua versatilità, semplicità e alla grande disponibilità di documentazione e risorse, data la sua natura \textit{open source}.\\
Si è visto tuttavia come in determinati casi questa potrebbe non essere la scelta migliore.\\
Allo stesso tempo, fare affidamento solo alle tecnologie già consolidate, con alle spalle anni di sviluppo e di esperienza, può ancorare il proprio software a delle soluzioni vecchie o inadeguate.\\

\noindent Anche in questo caso, la soluzione migliore deriva da un compromesso. L'utilizzo di database ibridi che permettono di sfruttare il meglio di entrambe le tecnologie può garantire migliori prestazioni senza sacrificare solidità delle strutture e affidabilità del servizio.\\


%**************************************************************
\section{Risultati specifici per il contesto di Ifin Sistemi}
Per quanto riguarda l'azienda per cui è stato condotto lo studio e il software su cui è stato basato, i risultati suggeriscono l'implementazione di una soluzione ibrida, che potrebbe migliorare il software, alleggerendo il carico di lavoro per i database tradizionali e permettendo di implementare nuove funzionalità.\\

\noindent Non è tuttavia prevista, nell'immediato futuro, un'analisi più approfondita volta a provare quanto postulato con questa tesi, per rendere più realistica l'idea di una migrazione di database. Come già evidenziato dal dialogo avuto con i team dedicati allo sviluppo dei software analizzati, le soluzioni adottate fin'ora per potenziare i database in uso sono state al contempo dispendiose da implementare (a livello di tempo ed energia) e proficue nel modo in cui sono riuscite a tamponare i problemi di disponibilità a cui tali tecnologie sono andate incontro con l'aumentare del carico di lavoro imposto.\\
Finchè i sistemi adottati continuano a funzionare stabilmente, non ha senso per l'azienda dedicare forza lavoro allo studio di nuove tecnologie, soprattutto in luce di un secondo fattore estremamente centrale: i prodotti dell'azienda non sono nuovi. Questo ha un peso per quel che riguarda la necessità di far ``collaborare'' tecnologie nuove con altre meno recenti, ma significa soprattutto che negli anni in cui \textit{LegalArchive} e \textit{InvoiceChannel} (i prodotti in questione) sono stati in commercio, il naturale ciclo di manutenzione e upgrade del software ha avuto corso, rendendoli dei progetti estremamente grandi e complessi, la cui migrazione richiederebbe probabilmente più energie di quelle che si andrebbero a risparmiare una volta terminata.\\

\noindent Ciò che questa ricerca può aspirare ad essere è uno spunto per evidenziare quanto l'implementazione ibrida di soluzioni relazionali miste a quelle NoSQL possa essere proficua. La soluzione che viene proposta nella \autoref{sec:fattura-integrale} è un esempio valido di questa convivenza di tecnologie, e rappresenta un buon punto di partenza qualora l'azienda decidesse di percorrere questa strada.\\

%**************************************************************
\section{Raggiungimento degli obiettivi formativi}

Il raggiungimento degli obiettivi è riassunto nella \autoref{tab:tracciamento}. Nonostante le leggere variazioni subite dal piano di lavoro, tutti gli obiettivi obbligatori sono stati quasi tutti raggiunti durante lo svolgimento del tirocinio.\\
L'unica eccezione viene fatta per l'obiettivo \texttt{O06}, a causa di leggere variazioni nelle attività svolte. Mentre l'obiettivo prevedeva la creazione di una suite di test di carico, a causa dei tempi ristretti e delle reali necessità riscontrate durante lo svolgimento dello stage si è ritenuto più consono limitarsi alla creazione di singole query che fossero in grado di mettere meglio a confronto i database analizzati.\\
Per quanto riguarda l'obiettivo \texttt{O07}, legato alla stesura di documentazione riassuntiva sui risultati ottenuti dal confronto, questa è stata abbozzata e scritta nella sua interezza durante il periodo di stage, mentre la  versione più formale è costituita da questa tesi stessa.\\
Anche l'unico obiettivo identificato come desiderabile è stato raggiunto, nello specifico durante lo studio delle tecnologie appartenenti all'ambito NoSQL.

\begin{table}
\begin{center}
    
    \renewcommand{\arraystretch}{1.5}
    
    \centering
    \begin{longtable}{| C{2cm} | C{7cm} | C{3.5cm} |}
        
        \hline
        
        \rowcolor{mongogreen}
        \textbf{Obiettivo} & \textbf{Descrizione} & \textbf{Stato} \\
        
        \hline
        
        \texttt{O01} & Comprendere lo stack di sviluppo aziendale & Raggiunto \\
        
        \hline 
        
        \texttt{O02} & Colmare eventuali lacune sullo stack Java EE & Raggiunto \\
        
        \hline 
        
        \texttt{O03} & Colmare eventuali lacune in ambito NoSQL & Raggiunto \\
        
        \hline 
        
        \texttt{O04} & Identificare i KPI aziendali & Raggiunto \\
        
        \hline
        
        \texttt{O05} & Ottenere una visualizzazione chiara del panorama NoSQL & Raggiunto  \\
        
        \hline 
        
        \texttt{O06} & Suite di test di carico e verifica performance e KPI & Parzialmente raggiunto \\
        
        \hline 
        
        \texttt{O07} & Esecuzione test e stesura documentazione riassuntiva & Raggiunto  \\
        
        \hline 
        
        \texttt{D01} & dentificazione librerie Java di connessione & Raggiunto  \\
        
        \hline 
        
        \caption{Tracciamento dei requisiti}\label{tab:tracciamento}
    \end{longtable}
    
    
\end{center}
\end{table}

%**************************************************************
\section{Valutazione retrospettiva delle attività svolte}
In \autoref{tab:retrospettiva} vengono riportare le attività effettivamente svolte durante lo stage e le ore impiegate. Come già detto nel \autoref{cap:descrizione-stage}, questa retrospettiva si discosta leggermente dalla pianificazione effettuata. Si è infatti rivelato necessario investire 40 ore nello studio del database di \textit{InvoiceChannel}, per poter costruire i due database di test su cui eseguire le query.\\
Questo quantitativo di ore, spese nella sesta settimana, non ha aumentato il monte ore previsto per il tirocinio, poichè sono state sottratte al tempo necessario per creare le query, che hanno richiesto meno tempo del previsto (attività inizialmente prevista per la sesta e la settima settimana).
Gli obiettivi delle ultime due settimane, inoltre, sono stati corretti in ``creazione, codifica e revisione delle query'', e non più dei test. Questo rispecchia il lavoro svolto, che non ha compreso la creazione di test di carico elaborati, ma si è limitato appunto all'elaborazione di query specifiche per PostgreSQL e MongoDB.


\begin{table}
\begin{center}

        \renewcommand{\arraystretch}{1.5}
    
        \centering
        \begin{longtable}{| C{2.5cm} | C{2cm} | L{7.2cm} | }
            
            \hline
            
            \rowcolor{mongogreen}
            \textbf{Durata in ore} & \textbf{Settimana} & \textbf{Descrizione} \\
            
            \hline
            
            40 & 1 &
            \begin{itemize}[leftmargin=*]
                \item Formazione sullo stack operativo e di sviluppo aziendale;
                \item Formazione stack \textit{Java EE}
            \end{itemize} \\
            
            \hline
            
            80 & 2, 3 &
            \begin{itemize}[leftmargin=*]
                \item Studio NoSQL in generale;
                \item Verifica soluzioni NoSQL specifiche;
                \item Identificazione di soluzioni NoSQL enterprise da analizzare e \gls{KPI}\ped{G} aziendali. 
            \end{itemize}  \\
            
            \hline
        
            80 & 4, 5 &
            \begin{itemize}[leftmargin=*]
                \item Analisi di dettaglio delle soluzioni precedentemente identificate. 
            \end{itemize}  \\
             
            \hline
            
            40 & 6 &
            \begin{itemize}[leftmargin=*]
                \item Studio del database aziendale e preparazione di una sua versione semplificata.
            \end{itemize}  \\

            \hline

            40 & 7 &
            \begin{itemize}[leftmargin=*]
                \item Creazione e codifica di query per lo studio delle performance di carico e aderenza ai \gls{KPI}\ped{G} individuati.
            \end{itemize}  \\
            
            \hline
            
            40 & 8 &
            \begin{itemize}[leftmargin=*]
                \item Revisione query e stesura documentazione finale.
            \end{itemize} \\
            
            \hline
            
            \rowcolor{mongogreen}
            \multicolumn{2}{ | c | }{\textbf{Totale ore: }} &   \multicolumn{1}{  c | }{\textbf{320}}\\
            
            \hline
        
            
            \caption{Attività svolte}\label{tab:retrospettiva}
        \end{longtable}
        
    
\end{center}
\end{table}

%**************************************************************
\section{Conoscenze acquisite e valutazione personale}
Il progetto di tirocinio mi aveva inizialmente attirato per la prospettiva di riprendere in mano le consocenze acquisite nell'ambito delle basi di dati ed approfondirle entrando a contatto con tecnologie più nuove, in grado di mettere in discussione i paradigmi su cui ci si basa quando si approccia per la prima volta questo contesto.\\

\noindent Le mie aspettative sono state soddisfatte ampiamente, sia per quanto riguarda la possibilità di eseguire degli studi su quello che è lo stato dell'arte ad oggi nell'ambito NoSQL, ma anche per le possibilità che ho avuto di sviluppare qualcosa di più concreto e potenzialmente utile, sia per me che per l'azienda.\\

\noindent È stato inoltre estremamente interessante ed importante per me avere la possibilità di approcciarmi all'ambiente lavorativo, in cui ho potuto imparare a muovermi seguendo ritmi e direttive diverse da quelle a cui ci si abitua all'interno dell'Università.\\

\noindent La concomitanza di tutte queste condizioni ha reso questo stage un'esperienza proficua e nel complesso positiva.\\
