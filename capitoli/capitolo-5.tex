% !TEX encoding = UTF-8
% !TEX TS-program = pdflatex
% !TEX root = ../tesi.tex

%**************************************************************
\chapter{Progettazione dei database per il confronto tra tecnologie}
\label{cap:progettazione}
%**************************************************************

L'idea generale per effettuare il confronto tra le prestazioni di database relazionali e database NoSQL è di creare un \gls{Proof of Concept}\ped{G} in grado di gestire i dati di uno dei software di punta dell'azienda (o comunque dati molto simili ad essi per quanto riguarda mole e contenuto), e di fare questo lavoro per entrambi i tipi di database. Una volta fatto ciò, si potranno elaborare delle query che mettano a confronto le prestazioni.\\

\noindent Viste le considerazioni fatte sui due software ``fratelli'', ossia \textit{LegalArchive} e \textit{InvoiceChannel}, e considerato il risultato dei dialoghi avuti con i team leader di entrambi i progetti, la scelta è ricaduta su \textit{InvoiceChannel}. Il motivo principale di questa scelta è che sebbene i due software siano piuttosto simili nella loro parte di \textit{backend}, legata ai database, \textit{InvoiceChannel} è un software leggermente più recente e per questo meno convoluto di \textit{LegalArchive}.\\
Sebbene questi siano progetti commerciali, dotati di estensiva documentazione, è bene considerare che nel tempo allocato per effetturare lo stage non sarebbe stato possibile approfondire nel dettaglio tutti gli aspetti di questi software.\\
Per lo stesso motivo, anche una volta scelto \textit{InvoiceChannel}, è stato necessario condurre uno studio per identificare una piccola parte del database che esso utilizza, su cui basarsi per costruire i database necessari ai test di confronto.\\


%**************************************************************
\section{Modellazione del database relazionale}
Una volta definite queste premesse, è stato effettuato uno studio approfondito della struttura del database di \textit{InvoiceChannel}, ricostruendo le relazioni tra tabelle a partire da un file in formato \texttt{SQL} fornito dal team di test.\\
Esso contiene più di ottantamila righe di codice e definisce la struttura, il contenuto e le relazioni tra le circa 150 tabelle che compongono un database utilizzato per effettuare test di vario tipo all'interno dell'azienda.\\

\noindent Risulta ovvio come \textit{InvoiceChannel} sia troppo grande e complesso per poter essere rimodellato nella sua interezza in questo contesto. Anche pensare di utilizzarne soltanto una sezione risulta piuttosto complesso, poichè tutte le tabelle sono estremamente interconesse.\\
Un ulteriore livello di complessità viene posto dalle query, anch'esse molto convolute, che fanno spesso un uso massiccio di operazioni di \textit{join}.\\
L'opzione migliore è quindi ricostruire effettivamente un pezzetto di database di \textit{InvoiceChannel} ``ad-hoc'', per soddisfare le necessità del progetto, sia in formato RDBMS, sia in formato NoSQL.\\

\subsection{Definizione di tabelle e relazioni}
La sezione in questione è quella relativa alle fatture, che risulta piuttosto centrale allo scopo del software. Le tabelle coinvolte nel diagramma ER (entità-relazione), visibile in \autoref{fig:schema-er}, sono quelle che si occupano di immagazzinare i dati relativi alle fatture elettroniche, lo stato dei processi che le coinvolgono, i documenti ad essa collegati e l'ente che l'ha generata.\\

\begin{figure}[htbp]
\begin{center}
\includegraphics[height=22em]{immagini/ER-Mock-IC.png}
\caption{Diagramma per la sezione semplificata di \textit{InvoiceChannel}}
\label{fig:schema-er}
\end{center}
\end{figure}

\noindent Dopo aver identificato le tabelle su cui concentrarsi per condurre lo studio e le relazioni che intercorrono fra di esse, si è passati alla definizione dei campi che compongono le suddette tabelle. Esse infatti devono essere ridimensionate per rispecchiare la sezione di database ridotto che è stata ricostruita.\\

\subsection{Script ridotti per la creazione delle tabelle}
Viene di seguito riportato il contenuto delle tabelle ristrutturate.\\

\begin{lstlisting}[language=SQL,
        deletekeywords={IDENTITY,INT},
        morekeywords={clustered},    
        framesep=10pt,
        framextopmargin=10pt, style=sql_style]
CREATE TABLE INVOICE(
    ID numeric(22, 0) NOT NULL,
    ID_OPERATION numeric(22, 0) NOT NULL,
    INVOICE_NUMBER Varchar(60) NULL,
    INVOICE_DATE Timestamp(3) NULL,
    TOTAL_AMOUNT numeric(25, 9) NULL,
    RECIPIENT_NAME varchar(255) NULL,
    DATA_INVIO Timestamp(3) NULL,
    SOURCE_FILE_NAME Varchar(255) NULL,
    TIPO_FATTURA Varchar(6) NOT NULL,
    DATA_RICEZIONE Timestamp(3) NULL,
    CAUSALE Varchar(200) NULL,
    MITTENTE Varchar(200) NULL,
    ID_ORGANIZATION numeric(22, 0) NULL,
    IS_B2B numeric(1, 0) NULL,
    TIPO_DOCUMENTO char(4) NULL,
    RIEPILOGO_IVA Text NULL,
 CONSTRAINT PK_INVOICE PRIMARY KEY ( ID )
);
\end{lstlisting}
\noindent La tabella \texttt{INVOICE} rappresenta la fattura, con tutti i dati e i codici che essa contiene. Nella sua versione ridotta sono stati eliminati molti campi che facevano riferimento a tabelle non più collegate o dati specifici appartenenti all'ambito della fatturazione elettronica non ritenuti necessari agli scopi di questa tesi.\\

\begin{lstlisting}[language=SQL,
        deletekeywords={IDENTITY,INT},
        morekeywords={clustered},    
        framesep=10pt,
        framextopmargin=10pt, style=sql_style]
CREATE TABLE DOCUMENT(
    ID numeric(22, 0) NOT NULL,
    ID_OPERATION numeric(22, 0) NOT NULL,
    CREATION_DATE Timestamp(3) NOT NULL,
    COD_DOCUMENT Varchar(255) NOT NULL,
    PATH_FILE varchar(1024) NULL,
    ID_REPOSITORY numeric(5, 0) NULL,
    DOCUMENT_SIZE numeric(22, 0) NULL,
    ID_ORGANIZATION numeric(22, 0) NULL,
 CONSTRAINT PK_DOCUMENT PRIMARY KEY ( ID )
);
\end{lstlisting}
\noindent La tabella \texttt{DOCUMENT} contiene una serie di dati relativi al contenuto di un documento. Ogni elemento della tabella \texttt{INVOICE} fa riferimento infatti ad uno o più documenti, tra cui la fatura vera e propria e potenzialmente altri allegati\\

\begin{lstlisting}[language=SQL,
        deletekeywords={IDENTITY,INT},
        morekeywords={clustered},    
        framesep=10pt,
        framextopmargin=10pt, style=sql_style]
CREATE TABLE OPERATIONS(
    ID numeric(22, 0) NOT NULL,
    CREATION_DATE Timestamp(3) NOT NULL,
    MESSAGE Text NULL,
 CONSTRAINT PK_OPERATIONS PRIMARY KEY ( ID )
);
\end{lstlisting}
\noindent La tabella \texttt{OPERATIONS} rappresenta lo stato di processo in cui si trova fattura all'interno del sistema. Man mano che il software processa la fattura, il suo stato viene aggiornato all'interno di questa tabella.\\

\begin{lstlisting}[language=SQL,
        deletekeywords={IDENTITY,INT},
        morekeywords={clustered},    
        framesep=10pt,
        framextopmargin=10pt, style=sql_style]
CREATE TABLE ORGANIZATION(
    ID numeric(22, 0) NOT NULL,
    NAME Varchar(255) NOT NULL,
    ALIAS varchar(100) NULL,
    PARTITA_IVA Varchar(255) NULL,
    COD_ORGANIZATION_TYPE Varchar(10) NULL,
    TIPO_SOCIETA Varchar(10) NULL,
    CREATION_DATE Timestamp(3) NULL,
 CONSTRAINT PK_ORGANIZATION PRIMARY KEY ( ID )
);
\end{lstlisting}
\noindent La tabella \texttt{ORGANIZATION} contiene informazioni riguardanti gli enti e le aziende che fanno uso del software, per poter tracciare la provenienza di una fattura e compiere moltissime altre operazioni all'interno dell'applicativo.\\

\begin{lstlisting}[language=SQL,
        deletekeywords={IDENTITY,INT},
        morekeywords={clustered},    
        framesep=10pt,
        framextopmargin=10pt, style=sql_style]
CREATE TABLE REPOSITORY(
    ID numeric(5, 0) NOT NULL,
    DEFINITION Varchar(255) NULL,
    COD_REPOSITORY_TYPE varchar(10) NULL,
 CONSTRAINT PK_REPOSITORY PRIMARY KEY ( ID )
);
\end{lstlisting}
\noindent La tabella \texttt{REPOSITORY} contiene informazioni riguardanti la localizzazione degli archivi che contengono i documenti. \\


%**************************************************************
\section{Modellazione del database in MongoDB}
In seguito alla costruzione del database relazionale che semplifica \textit{InvoiceChannel}, si può iniziare a costruirne la controparte NoSQL seguendo il modello documentale per implementarla con MongoDB.\\

\subsection{Relazioni tra tabelle}
Sebbene sia un modello documentale, Mongo mantiene comunque la capacità di mettere in relazione documenti provenienti da collezioni diverse. Bisogna tuttavia farlo nel modo corretto perché questo porti dei vantaggi.\\

\noindent Le possibili relazioni esistenti sono tre:
\begin{itemize}
    \item One to One
    \item One to Many
    \item Many to Many
\end{itemize}
Esiste inoltre un altro concetto proposto da mongo per rappresentare moli di dati enormi: si tratta dello ``zillion'', per indicare un ``many'' ancora più grande. Le vediamo con ordine, facendo riferimento a \cite{site:mongoDataModeling} per la trattazione qui proposta.

\subsubsection{One to One}
Sono le relazioni che intercorrono tra campi della stessa tabella, ma vengono usate anche per connettere tabelle diverse, per separare i dati secondo una determinata logica.\\
In MongoDB è possibile mantenere separati documenti che sono in relazione \textit{one to one}, usando quindi degli identificatori, oppure è possibile incorporare i documenti uno nell'altro (\textit{embedding}).\\
Riassumendo:
\begin{itemize}
    \item quando possibile, è meglio incorporare, per non introdurre complessità inutile.
\end{itemize}

\subsubsection{One to Many}
Sono relazioni in cui un'entità fa riferimento a molteplici altre entità, mentre queste altre possono riferirne soltanto una.\\
Anche in questo caso ci sono due possibili approcci, o si fa \textit{embedding}, o si mantengono le relazioni tramite indici.\\
Nel primo caso si può inserire il documento singolo in ogni elemento del gruppo di documenti riferiti, o in alternativa si può inserire nel singolo documento tutto il gruppo con cui era relazionato. Solitamente si fa \textit{embedding} nell'entità che subisce più query.\\
Se si decide invece di gestire la relazione con un indice, si può inserire una reference a tale indice, nel lato ``one'' o nel lato ``many''. Solitamente si sceglie il lato ``many''.\\
Riassumendo:
\begin{itemize}
    \item quando possibile, è meglio preferire l'operazione di \textit{embedding}, per favorire la semplicità, soprattutto se si può fare sulla collection maggiormente colpita da query;
    \item va bene usare reference quando si mettono in relazione documenti che non hanno lo stesso grado di query associate.
\end{itemize}

\subsubsection{Many to Many}
Le relazioni ``molti a molti'' sono solitamente più complesse, ma possono/devono essere semplificate nel modello relazionale introducendo una tabella di congiunzione in modo da spezzare la relazione in due relazioni ``molti a uno'';
Nel modello documentale questo non è strettamente necessario, ma è sempre giusto chiedersi se la relazione sia fondamentale così com'è o se si può semplificare.
Riassumendo:
\begin{itemize}
    \item preferire \textit{embedding} nel lato che subisce più query;
    \item preferire \textit{embedding} per informazioni che sono primariamente statiche e che possono funzionare bene anche in caso di duplicazione;
    \item preferire invece l'uso di reference per evitare di dover avere a che fare con la duplicazione, quando questo è un effetto indesiderato.
\end{itemize}

\subsubsection{Zillions}
Con questo termine si vuole identificare un subset delle relazioni \textit{one to many}, in cui ``many'' è significativamente grande. Serve un occhio di riguardo per evitare operazioni che recuperano tutti gli elementi coinvolti nella relazione, poiché questo andrebbe ovviamente a creare dei grossi rallentamenti.
Per questo motivo, in caso di relazioni fatte così, non è possibile pensare a soluzioni incorporate, e si può solo mantenere la relazione inserendo la reference nel lato ``zillions''.

%**************************************************************
\section{Utilizzo di pattern in MongoDB}
Un po' come i \gls{design pattern}\ped{G} nella programmazione ad oggetti, i pattern di modellazione dei dati sono utili per rendere questi modelli più comprensibili, basandosi su principi provati.\\
Per quanto riguarda questo progetto, andremo inizialmente a verificare quali sono i pattern consolidati e utili per lavorare con MongoDB, e successivamente determineremo se sarà consono applicarne qualcuno. Non avrebbe senso infatti applicarli a prescindere, se il progetto si rivelasse essere già abbastanza semplice così com'è.\\

\subsection{Potenziali problemi}
Prima di tutto, è necessario quindi specificare che l'utilizzo di pattern può causare l'introduzione di \textit{duplication}, \textit{staleness} e \textit{integrity issues} nei dati interni al modello. Questi possono essere un deterrente, specialmente nel caso di progetti semplici, ma se affrontati nel modo giusto rendono l'utilizzo dei pattern una scelta proficua.\\

\subsubsection{Primo Problema: Duplication}
Può essere il risultato dell'inserimento di documenti all'interno di altri documenti, per velocizzare le query.\\
Ci sono casi in cui la duplicazione può essere desiderabile. Per esempio, se si fa \textit{embedding} dell'indirizzo di spedizione nell'ordine di un prodotto, questo sarà una copia dell'indirizzo dell'utente. Tuttavia l'informazione è statica e anche se l'utente cambiasse indirizzo in futuro questo non influenzerebbe gli ordini passati e già ricevuti.\\
Questo porta ad una soluzione migliore rispetto al dover referenziare l'indirizzo, cosa che diventerebbe lenta e non necessaria per lo scopo dell'applicativo.\\

\noindent Altro caso è quello della duplicazione di informazioni il cui peso in memoria è talmente piccolo che non provoca problemi. Se, per esempio, all'interno di un database dedicato all'archiviazione di informazioni relative al mondo del cinema, si decidesse di incorporare le informazioni riguardanti gli attori all'interno dei dati relativi ai film in cui recitano, questa sarebbe una soluzione accettabile, poiché si tratterebbe di informazioni statiche che non andrebbero cambiate. Rimane una soluzione migliore rispetto al dover gestire una relazione \textit{many to many} tra attori e film in cui compaiono.\\

\noindent Ultimo caso, molto più importante, è quello in cui la duplicazione crea problemi.\\
Questo è spesso il caso di dati che devono essere aggiornati nel tempo. Avere più copie di questi dati porta ad un \textit{overhead} di lavoro per mantenerli coerenti. In questi casi è necessario valutare tale \textit{overhead} a confronto con una soluzione che non usa il pattern in questione (che sta provocando la necessità di duplicazione), per capire quale soluzione adottare.\\

\subsubsection{Secondo Problema: Staleness}
Con questo termine si intende l'obsolescenza dei dati, che vengono quindi aggiornati troppo raramente. Fornire dei dati obsoleti può creare problemi in una moltitudine di situazioni.\\
Si tratta di un problema che affligge i sistemi moderni a causa della grande velocità con cui i dati vengono recuperati e aggiornati. A causa di ciò non si può avere la certezza di avere l'ultimo dato aggiornato.\\
La domanda da porsi quindi è: ``quanta obsolescenza è accettabile?''. Ovviamente dipende dal tipo di dato.\\
La soluzione è usare \textit{batch updates} in modo da avere la sicurezza di fare update in un tempo fissato, e per avere la possibilità di visionare gli update tramite uno stream.\\

\subsubsection{Terzo Problema: Referential Integrity}
Solitamente si hanno problemi con questo aspetto quando si eliminano parti di documenti o tabelle, senza eliminare le loro references. Non c'è quindi la possibilità di applicare un ``on cascade - delete'', che daremmo per scontata nei database relazionali. MongoDB non supporta questa funzione quindi è necessario che sia l'applicazione ad occuparsene attivamente.\\\\

\noindent Riassumendo quindi quanto visto, per ogni dato che intendiamo maneggiare dobbiamo chiederci quanto segue:
\begin{itemize}
    \item È necessario/utile duplicare quest'informazione?
    \begin{itemize}
        \item Risolvere usando \textit{batch updates}.
    \end{itemize}
    \item Qual'è il livello accettabile di \textit{staleness}?
    \begin{itemize}
        \item Risolvere con update basati sul \textit{\gls{change stream}}\ped{G}.
    \end{itemize}
    \item Quali parti del dato necessitano maggiormente di \textit{referential integrity}?
    \begin{itemize}
        \item Risolvere o prevenire implementando transazioni o \textit{\gls{change stream}}\ped{G}.
    \end{itemize}
\end{itemize}

\subsection{Possibili pattern}
Andiamo ora a vedere nello specifico quali sono i pattern che si possono applicare ad uno schema documentale, e quali sono i vantaggi che possono apportare, sempre facendo riferimento a \cite{site:mongoDataModeling}.

\subsubsection{Attribute Pattern}
In alcuni casi, nel costruire un documento, può sorgere la necessità di creare molti campi simili ma non identici (e quindi non appartenenti ad uno stesso array, per esempio). Se poi si volesse poter cercare informazioni all'interno di questi campi in modo simultaneo, questo diventa complesso.\\
Per risolvere questo problema si usa il pattern degli attributi. L'idea è di raggruppare in qualche modo questi campi che di fatto sono simili e contengono informazioni che ha senso mantenere raggruppate.\\

\begin{figure}[htbp]
\begin{center}
\includegraphics[height=15em]{immagini/attribute-pattern.png}
\caption{Esempio di utilizzo dell'attribute pattern}
\label{fig:attributepattern}
\end{center}
\end{figure}

\noindent In questo modo si possono inserire indici ed effettuare ricerche su \texttt{releases.v}, come si può vedere nell'esempio in \autoref{fig:attributepattern}.\\
I casi d'uso più comuni sono il raggruppamento di caratteristiche di un prodotto, o di un set di campi che hanno lo stesso tipo (come visto sopra per le date).

\subsubsection{Extended Reference Pattern}
Questo pattern è tra quelli che generano duplicazione nei dati, e va quindi usato con cura.
L'idea è di migliorare le operazioni che coinvolgono relazione \textit{One to Many}.\\
Immaginiamo di avere una relazione di quel tipo in cui, per collegare due documenti, è presente un id nell'entita ``One'' e una reference a tale id in ogni entità nel lato ``Many''. Se le operazioni che coinvolgono questi due documenti sono molto frequenti e richiedono delle informazioni che esistono solo nel lato ``Many'', saranno necessari molti \textit{join} che rallentano per forza di cose l'utilizzo del database.\\
Per ovviare a questo problema, si decide quindi di rendere la reference non soltanto un campo del documento ``Many'', ma espanderla in un sottodocumento che contenga il riferimento all'id e alcuni dei campi del documento ``One'', quelli più frequentemente utilizzati nelle query, in modo da evitare il grosso dei \textit{join}.\\
Tutto questo, come dicevamo, produce duplicazione di dati. Capiamo ora come gestirla.
Prima di tutto va minimizzata:
\begin{itemize}
    \item scegliendo di duplicare quei dati che non cambiano frequentemente;
    \item duplicando solo i dati necessari ad evitare le operazioni di \textit{join}.
\end{itemize}
\noindent Quando avviene un aggiornamento dell'originale:
\begin{itemize}
    \item capire quali sono le reference estese da aggiornare a loro volta;
    \item capire quando queste reference dovrebbero essere aggiornate.
\end{itemize}
\noindent È bene inoltre tenere a mente che a volte la duplicazione è desiderabile, rispetto all'utilizzo di una normale ``singola'' reference.

\subsubsection{Subset Pattern}
Anche questo pattern si occupa di migliorare le prestazioni del database, questa volta concentrandosi su problemi di \textit{working set}.\\
Quando l'insieme di dati su cui si deve lavorare è più grande dello spazio disponibile ad accesso veloce (per esempio in RAM), il continuo scambio di informazioni con la memoria diventa un collo di bottiglia.\\
Per ovviare a questo problema, notiamo innanzitutto come in determinati casi il \textit{working set} è sì troppo grande, ma è anche utilizzato solo parzialmente. Nei campi di un documento ``film'' troviamo la lista di tutto il cast, ma raramente vorremo visualizzarlo per intero. Ci basterà una lista dei 10 o 20 attori più importanti, nella maggior parte dei casi. Stessa cosa vale per le recensioni, e altri campi di questo tipo.\\
In questi casi risulta quindi utile creare un \textit{subset} di questi campi, in modo da includere solo la parte importante di queste grandi liste all'interno del \textit{working set}, e lasciare il resto in memoria per accedervi solo quelle rare volte che servirà, lasciando spazio a informazioni più importanti.\\
Si tende quindi a dividere le \textit{collections} in due segmenti, quello con i dati più importanti e quello con i dati che ricevono meno accessi.\\

\subsubsection{Computed Pattern}
Questo pattern si basa sulla comprensione del peso che le operazioni di lettura e scrittura hanno sul sistema che si sta utilizzando.\\
In particolare, quando vengono effettuate delle operazioni (somme o altri calcoli) in fase di lettura su un set di informazioni molto ampio, questo può rendere tale lettura estremamente lenta. L'idea è quindi di tenere traccia di tali operazioni (salvandone il risultato) per poi modificarle solo in fase di aggiunta di nuove informazioni, in modo che durante la lettura il dato sia già accessibile senza nuove operazioni necessarie.\\

\subsubsection{Bucket Pattern}
Il Bucket Pattern offre una via di mezzo tra dover contenere tutte le informazioni in un solo documento, e avere un documento per ogni informazione granulare.\\
Solitamente viene utilizzato per l'\gls{IoT}\ped{G} (\textit{Internet of Things}), dove grosse moli di dati vengono prodotte da sensori e rilevatori. Spesso si usa la data come discriminante per creare il \textit{bucket} in cui raggruppare le informazioni.\\
Ci sono anche degli svantaggi nell'introdurre questo pattern, legati al fatto che ora i dati sono separati in gruppetti, ed effettuare operazioni su tutti i dati diventa più complesso.\\

\subsubsection{Schema Versioning Pattern}
Questo pattern entra in gioco quando inevitabilmente arriviamo al punto di dover fare un upgrade alla struttura (\textit{schema}) dei documenti. Questo processo può essere dispendioso e complesso, soprattutto per migrare i dati già esistenti nel nuovo schema.\\
Il pattern propone di procedere come segue:
\begin{itemize}
    \item ogni documento riceve un campo \textit{schema-version} che identifica la versione dello schema su cui si basa, con cui è stato costruito;
    \item l'applicazione deve poter gestire tutte le versioni di schema presenti a database, per poter gestire il cambiamento in-itinere senza dover bloccare tutto durante la transizione ed effettuarla in un unico momento;
    \item si sceglie una strategia di migrazione dei dati, che sia essa aspettare l'update dei singoli documenti per altri motivi e sfruttare l'occasione per mutare anche lo schema, oppure fare dei \textit{batch update} appositamente dedicati a questa modifica.
\end{itemize}
\noindent Questo pattern è particolarmente utile quindi quando non ci si può permettere di avere \textit{downtime} sul sistema, che deve continuare a funzionare in modo costante.\\

\subsubsection{Tree Patterns}
Il modello documentale si presta bene per rappresentare modelli gerarchici di dati.\\
All'interno di questo tipo di rappresentazione, esistono alcuni pattern utili per migliorarne l'utilizzo. Come per gli altri pattern, anche in questo caso è importante capire quali siano le necessità del sistema che stiamo modellando, in modo da fare delle scelte accorte.\\
I pattern disponibili sono:
\begin{itemize}
    \item Parent References
    \item Child References
    \item Array of Ancestors
    \item Materialized Path
\end{itemize}
\noindent Ognuno ha vantaggi e svantaggi.\\
Nel caso di Parent References, si inserisce un campo ``genitore'' in ogni nodo, che indica appunto da quale nodo discende.\\
Nel caso di Child References è invece presente un array contenente tutti i ``figli immediati'', i nodi che sottostanno a quello corrente.\\
Array of Ancestors consiste in un campo (un array appunto) che contiene il padre del nodo corrente, il padre del padre, e via dicendo fino alla radice della struttura.\\
Materialized Path è leggermente diversa dagli altri come soluzione, ma consiste nel salvare come stringa un campo contenente gli antenati del nodo corrente separati con un \textit{value separator}, quindi per esempio il punto. In questo modo si facilitano le operazioni, che possono sfruttare quel campo in una \textit{regular expression}.\\
Ovviamente si possono applicare più pattern contemporaneamente, in base alle operazioni da svolgere, perchè ogni pattern funziona bene in determinati casi, ma può avere difficoltà in altri.\\

\subsubsection{Polymorphic Pattern}
È un pattern semplice che sta alla base di altri pattern visti in precedenza. Affronta il problema di accorpare oggetti simili ma diversi nella stessa collezione, tenendo traccia del tipo specifico di oggetto tramite un campo apposito.\\
Ci saranno quindi una serie di campi di base condivisi tra gli oggetti e altri campi specifici, a volte contenuti in subdocuments.\\

\subsubsection{Riepilogo}
È chiaro come in base al caso d'uso, l'utilizzo di un pattern può migliorare l'architettura che sta alla base del database. Come evidenziato in \autoref{fig:tabpattern}, è sempre bene tenere a mente che pattern diversi sono utili in contesti diversi.

\begin{figure}[htbp]
\begin{center}
\includegraphics[height=25em]{immagini/patterns-table.png}
\caption{Applicabilità dei pattern in base al caso d'uso}
\label{fig:tabpattern}
\end{center}
\end{figure}

\section{Migrazione del database relazionale nel modello NoSQL}
Il database relazionale costruito ed inserito all'interno di PostgreSQL è descritto dalla \autoref{fig:schema-er}. Per costruire l'architettura dei documenti che risiederanno all'interno del database NoSQL, vengono riassunti i contenuti di tale figura:\\
\begin{itemize}
    \item \texttt{INVOICE} contiene dei dati relativi alla fattura inserita dall'utente;
    \item \texttt{OPERATIONS} contiene un messaggio relativo allo stato di tale fattura;
    \item \texttt{DOCUMENT} contiene il path ai vari documenti legati alla fattura, come un file di notifica del SdI, la fattura stessa, allegati, ecc;
    \item \texttt{REPOSITORY} contiene i riferimenti ai repository utilizzati per il salvataggio dei documenti;
    \item \texttt{ORGANIZATION} contiene le informazioni relative agli enti e alle aziende che usano IC. Ogni fattura o documento ``appartiene'' ad una organizzazione.
\end{itemize}

\noindent Inoltre, per quanto riguarda le relazioni che intercorrono fra le tabelle, si ha quanto segue:
\begin{itemize}
    \item \texttt{INVOICE - OPERATIONS} (\textit{One to One})
    \item \texttt{INVOICE - ORGANIZATION} (\textit{One to Many})
    \item \texttt{DOCUMENT - OPERATIONS} (\textit{One to Many})
    \item \texttt{DOCUMENT - ORGANIZATION} (\textit{One to Many})
    \item \texttt{DOCUMENT - REPOSITORY} (\textit{One to Many})
\end{itemize}

\noindent Dallo studio svolto sui metodi di costruzione di database documentali, si evince che l'idea generale è di accorpare, laddove possibile, più tabelle in un unico documento.\\
Questo vale per tabelle in relazione \textit{One to Many}, ma soprattutto \textit{One to One}.\\
Sarebbe quindi il caso di accorpare \texttt{INVOICE} e \texttt{OPERATIONS}, ovvero le fatture assieme con le operazioni ad esse associate (sostanzialmente un messaggio sullo stato della fattura).\\

\noindent In base alle best practices proposte dal team di MongoDB, la soluzione più coerente per costruire il database scelto per il confronto prestazionale individua tre collection separate per \texttt{INVOICE}, \texttt{DOCUMENT} e \texttt{ORGANIZATION}, strutturate come in \autoref{fig:mongoer}.\\

\begin{figure}[htbp]
\begin{center}
\includegraphics[height=18em]{immagini/ER-Mongo-IC.png}
\caption{Diagramma per l'architettura del database documentale}
\label{fig:mongoer}
\end{center}
\end{figure}

\noindent A seguito dello studio condotto sui pattern esistenti per la costruzione di database di questo tipo, risulta chiaro come l'aggiunta di uno di essi sarebbe una forzatura. Alcuni sarebbero indubbiamente utili qualora questo tipo di migrazione dovesse essere effettuato per componenti più corpose di \textit{InvoiceChannel}, ma per quanto riguarda la parte selezionata per questa tesi, essa risulta già adeguata senza l'utilizzo di pattern.

\subsubsection{Schema del database NoSQL}
Come già detto in fase di presentazione di MongoDB, questo tipo di database documentale è spesso denominato \textit{schema-less}. Ciò significa che a differenza dei database relazionali, un documento all'interno di MongoDB non deve rispettare alcuno schema per strutturare i dati al proprio interno.\\
I dati sono salvati in formato \texttt{JSON}, sfruttando coppie di chiave valore che possono a loro volta contenere array e sotto-documenti.\\
Riportiamo di seguito un esempio di documento per ognuna delle \textit{collection} individuate precedentemente, per capire meglio quanto visto nel diagramma dell'architettura.\\

\noindent Esempio di documento all'interno della collection \texttt{INVOICE}:
\begin{lstlisting}[language=json,
        deletekeywords={IDENTITY,INT},
        morekeywords={clustered},    
        framesep=10pt,
        framextopmargin=10pt]
{
    "ID": "1",
    "INVOICE_NUMBER": "2",
    "INVOICE_DATE": "1986-05-16T17:36:51.514Z",
    "TOTAL_AMOUNT": "32006.83",
    "RECIPIENT_NAME": "Floor Holdings Inc",
    "DATA_INVIO": "2012-10-10T18:18:33.919Z",
    "SOURCE_FILE_NAME": "edt",
    "TIPO_FATTURA": "CA",
    "DATA_RICEZIONE": "1994-05-09T10:37:37.325Z",
    "CAUSALE": "blvd yoga bg contractor",
    "MITTENTE": "Distance Software GmbH",
    "ID_ORGANIZATION": "72",
    "IS_B2B": "false",
    "TIPO_DOCUMENTO": "TD04",
    "OPERATIONS": {
      "CREATION_DATE": "1985-03-05T20:05:37.400Z",
      "MESSAGE": "Stato fattura = Primo_Stato"
    }
}
\end{lstlisting}
\noindent Come si può notare, il documento \texttt{OPERATIONS} è inserito nel documento della invoice come sottodocumento. Ciò significa che alla chiave \texttt{OPERATIONS} corrisponde un documento innestato, al posto di un semplice valore.\\

\noindent Esempio di documento all'interno della collection \texttt{DOCUMENT}:
\begin{lstlisting}[language=json,
        deletekeywords={IDENTITY,INT},
        morekeywords={clustered},    
        framesep=10pt,
        framextopmargin=10pt]
{
    "ID": "1",
    "ID_INVOICE": "462",
    "CREATION_DATE": "1970-09-18 04:42:03.000",
    "COD_DOCUMENT": "1",
    "PATH_FILE": "2",
    "DOCUMENT_SIZE": "19375",
    "ID_ORGANIZATION": "1",
    "REPOSITORY":{
        "ID": 1,
        "DEFINITION": "/home/invoicechannel/ic3/repository/repository70",
        "COD_REPOSITORY_TYPE": "HDD"
    }
}
\end{lstlisting}
\noindent Anche in questo caso esiste per ogni documento di \texttt{DOCUMENT} un documento innestato che fa riferimento al \textit{repository} di archiviazione.\\

\noindent Esempio di documento all'interno della \textit{collection} \texttt{ORGANIZATION}.
\begin{lstlisting}[language=json,
        deletekeywords={IDENTITY,INT},
        morekeywords={clustered},    
        framesep=10pt,
        framextopmargin=10pt]
{
    "ID": "1",
    "NAME": "Kuphal, Rath and O'Keefe",
    "ALIAS": "1",
    "PARTITA_IVA": "4",
    "COD_ORGANIZATION_TYPE": " Azienda",
    "TIPO_SOCIETA": " Altro",
    "CREATION_DATE": "1974-01-14 08:44:54.000"
}
\end{lstlisting}
\noindent I documenti di \texttt{ORGANIZATION} sono piuttosto semplici e contengono, come nella versione relazionale, le informazioni riguardanti enti e aziende che interagiscono con il sistema.\\

