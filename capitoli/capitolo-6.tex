% !TEX encoding = UTF-8
% !TEX TS-program = pdflatex
% !TEX root = ../tesi.tex

%**************************************************************
\chapter{Conclusioni}
\label{cap:conclusioni}
%**************************************************************
\section{Conclusioni riguardanti il confronto in generale}
Prima di avviare questo progetto di tesi è stata effettuata una breve ricerca per comprendere se l'argomento fosse stato già esplorato in precedenza in altri contesti.\\
Sono state individuate altre tesi dai temi simili che propongono il confronto all'interno di database ``da laboratorio'', e alcune ricerche condotte su larga scala da aziende.\\
Vista la natura di questo progetto, legato ai prodotti software sviluppati da Ifin Sistemi, la prospettiva di effettuare una ricerca approfondita risultava comunque interessante.\\

\noindent Le conclusioni raggiunte al termine dei due mesi di tirocinio non sono dissimili da quelle raggiunte da altre ricerche, quando queste erano condotte senza lo scopo di pubblicizzare l'una o l'altra tecnologia.\\
Come in molti altri ambiti, anche in questo caso il contesto di utilizzo di uno strumento ne determinano efficacia ed efficienza.\\
L'introduzione dei database NoSQL ha permesso di raggiungere risultati estremamente vantaggiosi in molti ambiti applicativi, ma questo non li rende uno strumento universale per potenziare qualsiasi database in ogni contesto.\\
Tale fatto è valido per qualsiasi tecnologia, ma lo è ancora di più quando si parla di database NoSQL, data la grande varietà di soluzione che questo gruppo comprende.\\
Nel caso specifico di MongoDB è facile essere tentati di farne largo uso in molti contesti diversi, grazie alla sua versatilità, semplicità e alla grande disponibilità di documentazione e risorse, data la sua natura open source.\\
Si è visto tuttavia come in determinati casi questa potrebbe non essere la scelta migliore.\\
Allo stesso tempo, fare affidamento solo alle tecnologie già consolidate, con alle spalle anni di sviluppo e di esperienza, può ancorare il proprio software a delle soluzioni vecchie o inadeguate.\\

\noindent La soluzione migliore, come spesso accade, sta nel mezzo. L'utilizzo di database ibridi che permettono di sfruttare il meglio di entrambe le tecnologie può garantire migliori prestazioni senza sacrificare solidità delle strutture e affidabilità del servizio.\\


%**************************************************************
\section{Conclusioni specifiche per il contesto di Ifin Sistemi}
Per quanto riguarda l'azienda per cui è stato condotto lo studio e il software su cui è stato basato, i risultati sono evidenti. L'implementazione di una soluzione ibrida potrebbe giovare di molto al sistema, alleggerendo il carico di lavoro per i database tradizionali e permettendo di implementare nuove funzionalità.\\

\noindent Non è tuttavia prevista, nell'immediato futuro, un'analisi più approfondita volta a provare quanto postulato con questa tesi, per rendere più realistica l'idea di una migrazione di database. Come già evidenziato dal dialogo avuto con i team dedicati allo sviluppo dei software analizzati, le soluzioni adottate fin'ora per potenziare i database in uso sono state al contempo dispendiose da implementare (a livello di tempo ed energia) e profique nel modo in cui sono riuscite a tamponare i problemi di disponibilità a cui tali tecnologie sono andate incontro con l'aumentare del carico di lavoro imposto.\\
Finchè nulla si rompe, non avrebbe senso per l'azienda dedicare forza lavoro allo studio di nuove tecnologie, soprattutto in luce di un secondo fattore estremamente centrale: i prodotti dell'azienda non sono nuovi. Questo ha un peso per quel che riguarda la necessità di far comunicare tecnologie nuove con altre meno recenti, ma significa soprattutto che negli anni in cui LegalArchive e InvoiceChannel (i prodotti in questione) sono stati in commercio, il naturale ciclo di manutenzione e upgrade del software ha avuto corso, rendendoli dei progetti estremamente grandi e complessi, la cui migrazione richiederebbe probabilmente più energie di quelle che si andrebbero a risparmiare una volta terminata.\\

\noindent Ciò che questa ricerca può aspirare ad essere è uno spunto per evidenziare quanto l'implementazione ibrida di soluzioni relazionali miste a quelle NoSQL possa essere profiqua. La soluzione che viene proposta nella \autoref{sec:fattura-integrale} è un esempio valido di questa convivenza di tecnologie, e rappresenta un buon punto di partenza qualora l'azienda decidesse di percorrere questa strada.\\


%**************************************************************
\section{Conoscenze acquisite e valutazione personale}
Il progetto di tirocinio mi aveva attirato per la prospettiva di riprendere in mano le consocenze acquisite nell'ambito delle basi di dati ed approfondirle entrando a contatto con tecnologie più nuove, in grado di mettere in discussione i paradigmi su cui ci si basa quando si approccia per la prima volta questo ambito.\\

\noindent Le mie aspettative sono state soddisfatte ampiamente, sia per quanto riguarda la possibilità di eseguire degli studi su quello che è lo stato dell'arte ad oggi nell'ambito NoSQL, ma anche per le possibilità che ho avuto di sviluppare qualcosa di più concreto e potenzialmente utile, sia per me che per l'azienda.\\

\noindent È stato inoltre estremamente interessante ed importante per me avere la possibilità di approcciarmi all'ambiente lavorativo, in cui ho potuto imparare a muovermi seguendo ritmi e direttive diverse da quelle a cui ci si abitua all'interno dell'Università.\\

\noindent La concomitanza di tutte queste condizioni ha reso questo stage un'esperienza proficua e nel complesso positiva.\\