% !TEX encoding = UTF-8
% !TEX TS-program = pdflatex
% !TEX root = ../tesi.tex

%**************************************************************
% Sommario
%**************************************************************
\cleardoublepage
\phantomsection
\pdfbookmark{Sommario}{Sommario}
\begingroup
\let\clearpage\relax
\let\cleardoublepage\relax
\let\cleardoublepage\relax

\chapter*{Sommario}

Il presente documento descrive il lavoro svolto durante il periodo di stage svolto presso l'azienda Ifin Sistemi s.r.l. Lo stage è stato svolto alla conclusione del percorso di studi della laurea triennale in Informatica, occupando circa trecentoventi ore divise in otto settimane.\\
Lo scopo del progetto svolto è stato effettuare uno studio di fattibilità per l’integrazione di una soluzione di database NoSQL nei prodotti dell'azienda. Lo studio di fattibilità ha comportato una fase di analisi delle varie soluzioni NoSQL esistenti sul mercato, una fase di analisi delle soluzioni attualmente adottate all'interno dei prodotti Ifin, ed infine una fase di valutazione pratica delle soluzioni individuate, con relativi benchmark per il confronto delle prestazioni ed un approfondimento sulle differenze di progettazione tra database relazionali classici e database NoSQL.

\vfill

\chapter*{Convenzioni Tipografiche}

All'interno del presente documento verranno utilizzate le seguenti convenzioni tipografiche:
\begin{itemize}
    \item le espressioni appartenenti alla lingua inglese ed il nome di tecnologie e strumenti vengono evidenziati in \textit{corsivo} all'interno del testo;
    \item eventuali termini meno diffusi vengono inseriti all'interno del glossario, consultabile in fondo alla tesi e raggiungibile tramite link ipertestuale evidenziato in \textcolor{RoyalBlue}{blu}. Per ulteriore chiarezza, tutte le parole inserite nel glossario sono seguite da una ``G'' a pedice;
    \item le parole che si riferiscono ai frammenti di codice o al formato di file e documenti vengono evidenziate utilizzando lo stile \texttt{typewriter}.
\end{itemize}


%\vfill
%
%\selectlanguage{english}
%\pdfbookmark{Abstract}{Abstract}
%\chapter*{Abstract}
%
%\selectlanguage{italian}

\endgroup			

\vfill

